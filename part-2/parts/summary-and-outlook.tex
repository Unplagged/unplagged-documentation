\chapter{Project Analysis and Outlook}\label{chap:summaryAndOutlook}

To actually learn and take away as much as possible after nine month of development on such a big project like Unplagged we believe that thoroughly analyzing the course of the project and most importantly the mistakes is really important. The problem with this is always that there are some kind of \enquote{politics} involved, because we as a team and also every team member for themselves have an agenda when doing this. We are not graded yet, so we still want to shine with our achievements and not everybody in the team will feel the same about every topic that will or could be mentioned here. So this just as a disclaimer beforehand. 

We will nonetheless try to do our best in constructively critizing ourselves and the provided environment and look at what worked well in this chapter. Additonally we will give an outlook on possible improvements and further developments of the project.

\section{Project Analysis}

What was critized after the showtime by the professors was, that the overall progress of the software that was developed, wasn't quite enough for a group of five master students working nine month on it. We sadly have to admit, that this is at least kind of true, but we still can not concur fully with this assessment.

What we feel is that -- as was already said in the introduction -- we laid a nice foundation for an open source project with Unplagged and additionally learned a lot, which is something that probably can get overlooked easily, when simply looking at the end result. We understand, that it is difficult to grade those \enquote{soft factors}, but the goal for us as master students is not necessarily always a product that is just looking nice on the outside, but also one that uses \enquote{best-practices} on the inside. 
The difference between a well salted and encrypted password to one that is stored with simple text is nothing really visible, but something that makes a difference of a few hours in development after all.

\subsection{Using a Wiki}

One of the things that we were asked to do at the beginning was to maintain a Wiki, which we created inside our project management tool Redmine. Looking at the state of it now, we have to say, that this didn't work very well. 

We currently have 62 pages in there, with probably one third of them coming from meeting minutes and the rest simply being not very well written or up to date and therfore not useful, with just very few exceptions.

Starting it worked OK, but the information we put on there wasn't valid long enough in most instances, because our perception of the topics changed very fast in the beginning, which means the motivation and effort needed to keep the information relevant didn't match the usefulness. It was simply easier and even necessary after a while to ask the person that wrote the article or was currently working on that area of the system about it, because by the article you couldn't really tell if the information was still holding true. As we were just five people and mostly knew what the others worked on this was the path of least resistance and was adopted pretty quickly.

It also was problematic that we had just five editors for other reasons. As we could have put an endless amout of articles on the wiki, it often wasn't the case that someone wrote an article that then was corrected by someone else, but more that it stayed the way it was written, because the others were hopefully busy writing something about another topic, but maybe didn't even notice that there was something new. This lead to some kind of sprial of fewer and fewer edits, because when it didn't look like something was happening, the enticement of making edits was reduced further and further.

To make it work we believe that at least some good experiences of the eventual editors with a functioning wiki would be helpful, because with a group of just five people, everybody has to be dedicated to keep it running. Maybe making the decision to use a wiki more optional would also help, because it's simply harder to work on something, when the usefulness wasn't discovered but the necessity got kind of \enquote{enforced} from outside of the team.


\subsection{Agile development}

All in all the agile process we figured out after a while worked pretty good and is something that everyone of us feels more comfortable using now, so doing this helped us a lot experience wise. 

A problem we had with it is, that it took us really a while to pick up steam with this process that we never really used before. We are not quite sure if it also has something to do with the growing pressure to deliver something working in the end or most of us finally being comfortable with the development environment, but the sprints we did worked much more efficiently from the start of the second project semester.

What we feel is, that starting a bit earlier with coding could have helped the overall progress a lot, because the research of nearly one and a half month at the start where we didn't code, would probably have been more useful when we would have done it more along the way. Some starting points like registration and a basic setup of components could have been done right away and also would have been more in the mindset of an agile process.

\subsection{User Interface}

One of the points that was mentioned in the project description(see page \ref{cite:projectDescription}) was to put an emphasize on the usability and to test with \enquote{*Plaggers}, which is something we somehow neglected to do. It wasn't an intentional decision to not do usability testing, but rather something that we pushed farther and farther out in the schedule until we would finally feel more comfortable showing what we had achieved, but that never happened, because there were still some more features to be finished and some more ideas to be integrated.

We also only had one team member that really felt comfortable with designing a user interface in Photoshop, which is really a useful skill in web development, although wireframes like we did at the start would probably have been sufficient here, to plan a bit better. 

At some points we went back to review some parts of the interface, like the case selection dropdown for example, but new features were often times just added on the fly without a common process, so that would be something, we would really need to enforce for another project like this.

\subsection{Project Managment Tools}

The two main tools that we used to manage the project itself and it's code - Redmine and Git - were really a nice experience and worked pretty well for our purposes. They are both something we would recommend unconditionally and would use again without hesitation. 

Simply having a mechanism to do version control locally without being depended on a server with Git is such a nice feature, that some of us even adopted it for simple tasks like writing a letter.

\subsection{Unit Tests}

In the beginning we stated the goal to write the system in a test-driven development style, but sadly that never really happened. 

We believe this failed due to two key factors coming together. The first one is, that four of the five team members didn't have experience with developing test-driven and therefore never saw the nice benefits it can have after a while and the second is, that it would have been necessary to learn it in addition to the already overwhelming new development environment that we had set up.

\subsection{Reponsibilities}

The biggest problems we had often come down to unclear responsibilities. We defined some roles for the team members in the beginning, but we never emphasized them enough or gave the people that possessed them the power to enforce them. This probably happened because we didn't know each other at the start, so that we didn't know if we could trust the qualities of each others work in those areas for which the assignments were made. We also essentially had the feeling that we all should be on the same level, because we all simply were students, so no one wanted to step up and no hierarchy was clear prior to it, because we didn't knew each other for long enough time.

Overall this made us some kind of \enquote{doacracy}, where things got done in the way they were started by someone, which can work very well, but it ultimately lead to the problems described above with unit testing or the user interface for example.

\section{Outlook}

