\chapter{Project Overview}\label{chap:Overview}

The main project description we started to work with was quite short with about 300 words and was also hugely complemented and clarified in meetings with Prof. Weber-Wulff. Nonetheless it contains very important bits and pieces that are the foundation of what was built during the project:

\begin{quote}
\enquote{Right. I'm the plagiarism \enquote{huntress}. I've tested plagiarism detection software since 2004. Mostly, they suck. They either don't work, or are a pain to use, or both. I've spent 10 years trying to educate people about plagiarism, and still the media thinks I have a magic secret for discovering plagiarism, as demonstrated on the GuttenPlagWiki and the VroniPlagWiki. What actually happens there is that quite a number of tools are used for preparing texts, discovering possible sources, comparing them, and documenting them. The last part is done by hand and takes an enormous amount of time.

The idea is to set up a Plagiarism Detection Cockpit that integrates all sorts of bits and pieces, but leaves the teacher in command. It is not to be a general test system that spits out a number for every paper submitted, although the integration of as many such systems as possible will be one of the necessary features of the system. There will be a lot of thought needed for the interface design, as there are massive amounts of data that need to be displayed. How can this be compressed and fit on a screen? For example, a barcode-generation needs to be integrated.

The goal will be to provide a tool that easily produces simple-to-read documentation and deals with all sorts of nastiness that might turn up on the way. The tool must be multi-lingual and open source. It would also be cool to integrate some of the little tools such as the Android-based OCR-Scanner that was developed in the SS 2011. I would like for the team to use an agile development methodology so that we can continuiously test with teachers and professors and *Plaggers.

One of the first tasks will be collecting up the computing literature on the topic of plagiarism discovery. A wiki needs to be set up with links to material (online and offline), comments on the papers, and as many navigational indices as possible.}\citep{projectDescription}
\end{quote}


