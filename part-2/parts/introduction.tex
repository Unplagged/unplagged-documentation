\chapter*{Introduction}\addcontentsline{toc}{chapter}{Introduction}

Nine months after the start of development, the time of the masters project Unplagged at the HTW has now come to an end for the 
initial team members. Hopefully though, 
this isn't the end of the lifecycle of the software Unplagged that was created during that stretch. Some of us will try to continue working on it afterwards and maybe we are even able to attract some other developers to help this open source project in the long run if we are lucky. It was a useful experience after all and the foundation that was layed feels good and strong.

Although there are no direct connections, this present document is at least in some ways a sequel to the Developers Manual, that was
written after the first semester of this project. In some instances it assumes prior knowledge of things already described there, so that it doesn't unnecessarily need to be repeated here.
All in all, we are going to give a more technical insight into the system than in the report before, which concentrated mostly
on the project management and development environment aspects. We will also try to critically 
analyze the problems that were faced and the mistakes that were made over the course of those
two semesters.

For all the team members, the development of Unplagged up to this point was one of 
the biggest 
projects that we ever had to start and needed to bring to a state of usable 
maturity on our own. Most of us came across big or even huge projects at work some time, but the process until 
the first release was mostly long over and replaced by bugfixing and the eventual feature release.
So those team building phases forming, storming, norming and perfoming that Bruce Tuckman famously described\citep{tuckman1965},
were something we experienced in this project for the first time in full scale.

During our time of study, it was also one of the first projects, 
where we couldn't really envision in the beginning, how the end result would need to look like. It surely happened before at some points, but the implications of wrong decisions in the beginning weren't nearly as big then, as with such a long running development process like here.

With other projects before, there were also often requirements similar to some
parts of software systems we already knew and just needed to adapt to the current purpose.
This mostly gave us some hints of the feasibility and the best approach to solve a problem.
With Unplagged this was somehow different.

First of all, the idea for the project came from outside the team with very short time to research before the team building what it would be 
about, second we didn't have any experience with plagiarism research at all and additionally the team was bigger then we are normally used to. 
This means that we couldn't even imagine how the necessary workflow would need to look like, so we started out with one abstract idea of the system 
from outside of the development team, turned that into a different idea for every team member and eventually had to break it down to one software system.

We discoverd that Unplagged would have some similarities to project management systems like Redmine or Jira and maybe Google Docs after a while, but at the start this wasn't really obvious to us.

Luckily enough, the choices of technologies, architecture or programming languages we made in the beginning, didn't came to haunt us up to this point. Without the experiences we made at internships or work before, this probably would have been a different story though.

\section*{Chapter Overview}\addcontentsline{toc}{section}{Chapter Overview}

\begin{description}
\item[\ref{chap:Overview}. \nameref{chap:Overview}] \hfill \\
This first chapter will give a very brief overview of the preconditions and project structure, analyzing it by some key figures. It will also describe the \enquote{toolbox} of
frameworks and libraries that were integrated to have at our disposal for the development.
\item[\ref{chap:features}. \nameref{chap:features}] \hfill \\
Here we will describe the features and workflow implemented in the system so far.
\item[\ref{chap:Showtime}. \nameref{chap:Showtime}] \hfill \\
The chapter \nameref{chap:Showtime} will explain the preparations that were made for the presentation of the project results at the showtime.
\item[\ref{chap:summaryAndOutlook}. \nameref{chap:summaryAndOutlook}] \hfill \\
In the last chapter the main points are the discussion of the mistakes and
problems that were made during the development, with a focus on possible provisions against recurrence in future projects and an
outlook on the future of Unplagged.
\end{description}

\section*{Conventions}\addcontentsline{toc}{section}{Conventions}

To markup important words in the text, the following typographical conventions are used:

\begin{description}
\item \textit{Italic} \hfill \\
  First used technical terms
\item \texttt{Constant Width} \hfill \\
  Programm code, file names, paths
\item \textbf{\texttt{Bold Constant Width}} \hfill \\
  Variables that have to be changed by the user
\end{description}