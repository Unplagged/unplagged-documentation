\chapter*{Introduction and Overview}

Nine month after the start of development the time of the masters project Unplagged at the HTW has now come to an end for the 
initial team members. Hopefully though, 
this isn't the end of the lifecycle of the software Unplagged that was created during that stretch. Perhaps some of us will continue working on it afterwards or maybe we are even able to attract some other developers to help this open source project if we are lucky. It was a good experience after all and the foundation feels good and strong.

Although there are no direct connections, this document is at least in some ways a sequel to the Developers Manual, that was
written after the first semester of this project. In some instances it assumes prior knowledge of things already described there, so that it doesn't unnecessarily need to be repeated here.
All in all, we are going to give a more technical insight into the system then in the report before, which concentrated
on the project management and development environment aspects. We will also try to critically 
analyze the problems that were faced and the mistakes that were made over the course of those
two semesters.

For all the team members, the development of Unplagged up to this point was one of 
the biggest 
projects that we ever had to start and needed to bring to a state of usable 
maturity on our own. Most of us came across big or even huge projects at work some time, but the process until 
the first release was mostly long over and replaced by bugfixing and the eventual feature release.
So those famous team building phases forming, storming, norming and perfoming that Bruce Tuckman described\citep{tuckman1984},
were something we only could see here in a full scale.

During our time of study, it was also one of the first projects, 
where we couldn't really envision in the beginning, how the end result would need to look like.

With other projects before, there were often enough requirements similar to some
parts of software systems we already knew and just needed to adapt to the current purpose.
This mostly gave us some hints of the feasibility and the best approach to solve a problem.
With Unplagged this was somehow different.

First of all, the idea for the project came from outside the team with very short time to research beforehand what it would be 
about and second, we didn't have any experience with plagiarism research. So we started out with one abstract idea of the system 
outside of the development team, turned that into a different one for every team member and eventually had to break it down to one software system.

Luckily enough, the choices of technologies, architecture or even programming languages we made in the beginning, didn't came to haunt us in
the long run.
We eventually figured, that it would be some mix
